\label{chpt:conclusion} % for referencing this chapter elsewhere, use \ref{chpt:label}
\lhead{\emph{Conclusion and future work}} % This is for the header on each page - perhaps a shortened title

who the hell knows. What more can I say?

\subsection{Conclusion}

The component masses and radii, separations, white dwarf temperatures and surface gravities of 15 new short-period CVs are contributed to the population of well-characterised CV observations, including two with extremely low-mass donor stars, and one which appears to be in the process of evolving through the period minimum.
Some issues were encountered during the modelling of a handful of systems, which are recommended for UV spectroscopic followup studies to probe them in more detail: $T_{\rm eff}$ of the white dwarf in SSSJ0522-3505 appears to be $\sim$10000K higher than is typical for a CV, and the white dwarfs of CSS090419 and CSS090622 appear to \textit{brighten} in the $i'$ band, contrary to models.
% The derived temperature is quite uncertain, but the origin of the discrepancy cannot confidently be determined, though possible causes are given.
% All three of the newly modelled systems lie within $1\sigma$ of the ``optimal'' model mass-radius evolutionary tracks from \citet{knigge11}.

The ``optimal'' donor tracks add an extra source of AML that takes the form of $\sim 1.5$ times the GWB. By examining the period excess between the initial set of observed CV donor radii and models, this can be demonstrated to probably be inconsistent with the missing AML seen in observations.
Rather than tracking the GWB as the CV evolves to lower masses, excess AML appears to grow in strength relative to gravitational losses as the donor shrinks. These preliminary findings did {\it not} find evidence for a relationship between excess AML strength, and $M_{\rm wd}$.
These preliminary findings were corroborated by detailed evolutionary modelling; MESA was used to infer the $\dot M$ rate of a CV based on the mass loss-induced inflation measured using eclipse modelling, and a
