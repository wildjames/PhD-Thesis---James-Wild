\label{chpt:conclusion} % for referencing this chapter elsewhere, use \ref{chpt:label}
\lhead{\emph{Conclusions and future work}} % This is for the header on each page - perhaps a shortened title

I have extended the CV eclipse model to allow for a tiered structure of shared eclipse parameters as a logical progression of the work by \citet{McallisterThesis}.
Using this new hierarchical eclipse model I characterise the component masses and radii, separations, white dwarf temperatures and surface gravities of 15 new short-period CVs.
Future work should be directed to improving the optimisation of the eclipse light curve modelling -- the Affine invariant sampler with parallel tempering that is employed here is inefficient for problems with more than $\sim5$ parameters, and the model frequently has more than 100 free parameters.
While the optimisation works, it is likely sub-optimal. Unfortunately, the optimisation of expensive, non-differentiable models with many free parameters is inherently difficult, and it remains to be seen how significant an improvement is possible.

Some issues were encountered during the modelling of a handful of systems, which are recommended for UV spectroscopic follow-up studies to probe them in more detail: $T_{\rm eff}$ of the white dwarf in SSSJ0522-3505 appears to be $\sim$10000K higher than is typical for a CV, and the white dwarfs of CSS090419 and CSS090622 appear to \textit{brighten} in the $i'$ band, contrary to models.
However, I justify the assertion that these problems do not significantly impact the final results of modelling.

The newly extended sample of eclipse modelled CVs follows the canonical CV donor evolutionary tracks, though with significant scatter that suggests the possibility of a less unified donor track than is typically thought. A significantly larger sample is necessary before more concrete claims can be made.

I demonstrate the effectiveness of MESA in modelling CV evolution, and produce a configuration that replicates the canonical CV donor evolutionary sequence. I also note that with some additional work, MESA should be capable of even closer agreement.
Instead, I focus on an empirical reproduction of the observed M dwarf mass-radius relationship of \citet{brown2022} by introducing star spots to MESA. By adjusting the star spot parameters, I am able to exactly reproduce the Brown mass-radius relation.

Based on prior work by \citet{knigge2006,knigge11}, I use these MESA M dwarf models to deduce the mass loss rates of the M dwarf donors of eclipse modelled CVs by comparing their measured masses and radii with those of MESA models with varying degrees of mass loss.
As such, I present the first sample of donor-derived secular mass loss rate estimates for CVs, and use these data to interrogate the source of the long-standing excess AML implied by CV prior observations.
The data are preliminary, and are contingent on the validity of the M dwarf masses and radii to calibrate the MESA models.
The best available M dwarf mass-radius sample is sparsely populated below $\sim 0.12 M_\odot$, so I cannot marginalise over the intrinsic scatter in the M dwarf mass-radius relationship, biasing the sample towards higher $\dot M$ while also polluting it with some unphysically low $\dot M$ systems. Future observations of low mass M dwarf masses and radii should be able to rectify this shortfall.

The results here are considered preliminary for a few reasons. Primarily, the mass-radius relationship used to calibrate the MESA model radii at zero mass loss is poorly sampled and incomplete, reverting to theoretical models at $0.121 M_\odot$ -- the majority of my data lie in this poorly understood mass range.
In addition, significant further work is needed to grow the population of eclipse modelled low-$M_{\rm donor}$ CVs and improve these statistics, and continued eclipse modelling targeting short period CVs will be invaluable to confidently determining the source of excess AML.
Specific effort should be targeted towards confident characterisation of CVs at slightly higher $M_{\rm donor}$ of $\sim0.15$ (i.e. a period of $\sim 1.8 - 2.2$ hours), where existing data are somewhat sparse and have large uncertainties. The clustering of confident data at lower masses leaves the current sample subject to an over-reliance on data with donor masses below $\sim 0.12 M_\odot$.

These results suggest that magnetic braking is a poor description of the excess AML inferred from eclipse modelling of CVs, and indicate that eCAML continues to be a better descriptor of the data.
The basic prediction of eCAML -- that the relationship between $\dot J / J$ and $\dot M / (M_{\rm donor} M_{\rm wd})$ follows a straight line -- is seen in the data, though I suggest in \S\ref{sect:massloss and AML:systematic bias} that my $\dot M$ are systematically over-estimated so cannot yet be used to calibrate eCAML.
It must be reiterated, however, that the data presented here are themselves \textit{not} well-described by eCAML.
Calibrating the eCAML free parameter from these data results in a higher constant of proportionality and concludes that virtually all CVs are dynamically unstable, which is not self-consistent.

