\chapter*{Summary}

Cataclysmic variables are a type of close, interacting binary system, with a white dwarf primary and an M dwarf donor star that is in contact with its Roche lobe. As such, the outer layers of the M dwarf are gradually accreted onto the white dwarf, driven by angular momentum loss. Mass transfer and angular momentum loss dominates the evolution of these systems.

I characterise 15 new eclipsing cataclysmic variable stars, finding component masses and radii, and orbital separations by modelling their light curves in multiple filters. These characterisations conform to the results of previous similar works, tracking the canonical donor evolutionary sequence.

I develop a method to infer mass loss and angular momentum loss rates from donor properties. Stars are inflated by mass loss, and by replicating a donor star with the stellar evolutionary code, MESA, I can infer the mass loss rate the star is subjected to and calculate the corresponding angular momentum loss rate. I apply this method to the newly extended sample of eclipse-modelled cataclysmic variables, and report my findings.

The field of research around cataclysmic variables has struggled with an unknown contribution to angular momentum losses in the short period ($<2.5$ hours) regime for some time. This is seen in population synthesis models and evolutionary models, though discriminating between differing explanations for these excess losses has been somewhat challenging.
By comparing existing prescriptions for magnetic braking and consequential angular momentum loss (specifically, extra angular momentum loss resulting from successive nova eruptions) with observed mass loss and angular momentum loss rates, I present preliminary evidence in favour of nova eruptions being the dominant source of excess angular momentum losses.
These findings are limited primarily by the poorly understood and poorly characterised M dwarf mass-radius relationship, a problem likely to be mitigated with the release of Gaia DR3.